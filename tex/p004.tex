\upaper{4}{СВЯЗЬ БОГА СО ВСЕЛЕННОЙ}
\author{Божественный Советник}
\vs p004 0:1 Всеобщий Отец имеет вечный замысел, касающийся материальных, интеллектуальных и духовных явлений вселенной вселенных, который он выполняет на протяжении всего времени. Бог создал вселенные по своей собственной свободной и суверенной воле, и он создал их согласно своему премудрому и вечному замыслу. Сомнительно, что кто\hyp{}либо, кроме Райских Божеств и их высочайших партнёров, действительно очень много знает о вечном замысле Бога. Даже возвышенные граждане Рая придерживаются самых разных мнений о природе вечного замысла Божеств.
\vs p004 0:2 Легко сделать вывод, что целью создания совершенной центральной вселенной Хавоны было просто удовлетворение божественной природы. Хавона может служить образцовым творением для всех других вселенных и завершающей школой для паломников времени на их пути к Раю; тем не менее, такое небесное [supernal] творение должно существовать прежде всего для удовольствия и удовлетворения совершенных и бесконечных Создателей.
\vs p004 0:3 Изумительный план совершенствования эволюционных смертных и, после их достижения Рая и Корпуса Завершения, дальнейшего обучения для какой\hyp{}то нераскрытой будущей работы, представляется в настоящее время одной из главных забот семи сверхвселенных и их многочисленных подразделений; но эта схема восхождения для одухотворения и обучения смертных времени и пространства ни в коем случае не является исключительным занятием вселенских разумных существ. Действительно, есть много других увлекательных занятий, которые занимают время и задействуют энергию небесных воинств.
\usection{1.\bibnobreakspace ВСЕЛЕНСКАЯ ПОЗИЦИЯ ОТЦА}
\vs p004 1:1 Веками обитатели Урантии неправильно понимали провидение Бога. В вашем мире имеет место провидение божественного воздействия, но это не та детская, произвольная и материальная помощь, какой её представляли себе многие смертные. Провидение Бога состоит во взаимосвязанных действиях небесных существ и божественных духов, которые в соответствии с космическим законом непрестанно трудятся во славу Бога и для духовного продвижения его вселенских детей.
\vs p004 1:2 Разве ты не можешь продвинуться вперёд в твоём представлении об отношениях Бога с человеком до того уровня, где ты осознаешь, что девиз вселенной --- \bibemph{прогресс?} В течение долгих веков человеческий род боролся, чтобы достичь своего нынешнего положения. На протяжении всех этих тысячелетий Провидение разрабатывало план прогрессивной эволюции. Эти две мысли не противоречат друг другу на практике, а только в ошибочных представлениях человека. Божественное провидение никогда не препятствует истинному человеческому прогрессу --- ни мирскому [temporal], ни духовному. Провидение всегда соответствует неизменной и совершенной природе верховного Законодателя.
\vs p004 1:3 <<Бог верен>>, и <<все заповеди его справедливы>>. <<Его верность утверждена на самих небесах>>. <<На веки, Господи, твоё слово утверждено на небесах. Твоя верность --- всем поколениям; ты утвердил землю, и она пребывает>>. <<Он --- верный Создатель>>.
\vs p004 1:4 Нет ограничений на силы и личности, которые Отец может использовать для утверждения своего замысла и поддержки своих созданий. <<Вечный Бог --- наше прибежище, и под нами --- руки вечные>>. <<Тот, кто обитает в тайной обители Всевышнего, пребудет под тенью Всемогущего>>. <<Смотри, хранящий нас не воздремлет и не уснёт>>. <<Мы знаем, что всё работает вместе во благо любящих Бога>>, <<ибо очи Господа обращены к праведникам, а уши его открыты для их молитв>>.
\vs p004 1:5 Бог поддерживает <<всё словом силы своей>>. И когда рождаются новые миры, он <<посылает Сынов своих --- и они создаются>>. Бог не только творит, но и <<сохраняет их всех>>. Бог постоянно поддерживает всё творение материальное и всех существ духовных. Вселенные вечно устойчивы. Среди кажущейся неустойчивости существует стабильность. Среди энергетических потрясений и физических катаклизмов звёздных миров --- скрытый порядок и безопасность.
\vs p004 1:6 Всеобщий Отец не отстранился от управления вселенными; он не бездействующее Божество. Если Бог устранится, будучи вседержителем всего творения, то немедленно наступит всеобщий крах. Если бы не Бог, не было бы \bibemph{реальности.} В этот самый момент, так же, как и в отдалённые эпохи прошлого и в вечном будущем, Бог продолжает поддерживать. Божественная досягаемость простирается по кругу вечности. Вселенная не заведена, как часы, чтобы работать какое\hyp{}то время, а затем перестать функционировать; всё постоянно обновляется. Отец непрестанно изливает энергию, свет и жизнь. Работа Бога не только духовна, но и буквальна. <<Он простирает север над пустым пространством и подвешивает землю ни на чём>>.
\vs p004 1:7 \pc Существо моего порядка способно обнаружить предельную гармонию и заметить далеко идущую и глубокую координацию в повседневных делах вселенского управления. Многое из того, что смертному разуму кажется разрозненным и случайным, кажется мне упорядоченным и конструктивным. Но во вселенных происходит очень многое, чего я не могу полностью понять. Я давно изучаю и более или менее знаком с общеизвестными силами, энергиями, разумами, моронтиями, духами и личностями локальных вселенных и сверхвселенных. У меня есть общее понимание того, как действуют эти факторы и личности, и я близко знаком с работой аккредитованных духовных разумов большой вселенной. Несмотря на мои знания вселенских феноменов, я постоянно сталкиваюсь с космическими реакциями, которые я не могу полностью понять. Я непрерывно встречаюсь с кажущимся случайным слаженным взаимодействием сил, энергий, интеллектов и духов, которые я не могу удовлетворительно объяснить.
\vs p004 1:8 Я полностью компетентен, чтобы проследить и проанализировать действие всех феноменов, непосредственно проистекающих из функционирования Всеобщего Отца, Вечного Сына, Бесконечного Духа и, в значительной степени, Острова Рай. Моё недоумение вызвано столкновением с тем, что кажется действием их таинственных со\hyp{}управителей, трёх Абсолютов потенциальности. Эти Абсолюты, похоже, вытесняют материю, выходят за пределы разума и превосходят дух. Я постоянно нахожусь в растерянности и часто озадачен своей неспособностью понять эти сложные операции, которые я приписываю присутствиям и действиям Безусловного Абсолюта, Божества Абсолюта и Всеобщего Абсолюта.
\vs p004 1:9 Эти Абсолюты, должно быть, не полностью раскрытые широко во вселенной присутствия, которые в явлениях пространственной потенции и в функции других сверхпредельностей делают невозможным для физиков, философов или даже верующих [religionists] предсказывать с уверенностью, к\'ак именно первоисточники силы, идеи или духа будут реагировать на требования, предъявляемые в условиях сложной реальности, включающей верховные адаптации и предельные ценности.
\vs p004 1:10 \pc Во вселенных времени и пространства существует также органическое единство, которое, кажется, лежит в основе всей ткани космических событий. Это живое присутствие эволюционирующего Верховного Существа, эта Имманентность Спроецированного Незавершённого, необъяснимо проявляется время от времени тем, что, кажется, является удивительно случайной координацией явно не связанных между собой вселенских событий. Это, должно быть, и есть функция Провидения -- сфера Верховного Существа и Совместного Вершителя.
\vs p004 1:11 Я склонен верить, что именно этот обширный и, в целом, нераспознаваемый контроль координации и взаимосвязанности всех фаз и форм вселенской деятельности приводит к тому, что такая разнообразная и, казалось бы, безнадежно запутанная смесь физических, ментальных, моральных и духовных феноменов так безошибочно работвает во славу Бога и на благо людей и ангелов.
\vs p004 1:12 Но в более широком смысле, кажущиеся <<случайности>> космоса несомненно являются частью конечной драмы время\hyp{}пространственного приключения Бесконечного в его вечном манипулировании Абсолютами.
\usection{2.\bibnobreakspace БОГ И ПРИРОДА}
\vs p004 2:1 Природа --- это, в ограниченном смысле, физическая привычка [habit] Бога. Поведение или действие Бога определяется и условно модифицируется экспериментальными планами и эволюционными образцами локальной вселенной, созвездия, системы или планеты. Бог действует в соответствии с чётко определенным, неизменным, незыблемым законом по всей широко раскинувшейся главной вселенной; но он модифицирует шаблоны своего действия так, чтобы способствовать скоординированному и сбалансированному поведению каждой вселенной, созвездия, системы, планеты и личности в соответствии с локальными объектами, целями и планами конечных проектов эволюционного развития.
\vs p004 2:2 Следовательно, природа, как её понимает смертный человек, представляет лежащий в основании фундамент и основной фон для неизменного Божества и его незыблемых законов, модифицируемых, колеблющихся и переживающих потрясения в результате действия локальных планов, замыслов, образцов и условий, которые были введены в действие и выполняются локальной вселенной, созвездием, системой и планетарными силами и личностями. Например: законы Бога, установленные в Небадоне, были модифицированы планами, установленными Сыном Создателем и Созидательным Духом этой локальной вселенной; в дополнение ко всему этому, на действие этих законов дополнительно повлияли ошибки, нарушения и восстания определённых существ, проживающих на вашей планете и принадлежащих к вашей непосредственной планетарной системе Сатания\fnst{Слово <<Сат\'ания>>, возможно, происходит от древнееврейского \textheb{שָׂטָן} (сатан), означающего <<противник>>, что приводит к смыслу <<враждебная система>>. Но возможно и обратное: имя <<врага рода человеческого>> может само происходить от названия локальной системы и означать нечто совершенно иное и нам пока неизвестное.}.
\vs p004 2:3 \pc Природа является время\hyp{}пространственным результатом двух космических факторов: во\hyp{}первых, незыблемости, совершенства и прямоты Райского Божества, а во\hyp{}вторых, экспериментальных планов, грубых ошибок руководства, заблуждений мятежников, незавершённости развития и несовершенства мудрости вне\hyp{}Райских созданий, от высших до низших. Поэтому природа несёт в себе единообразную, неизменную, величественную и удивительную нить совершенства из круга вечности; но в каждой вселенной, на каждой планете и в каждой индивидуальной жизни эта природа видоизменяется, уточняется и, возможно, искажается действиями, ошибками и неверностью созданий эволюционных систем и вселенных; и поэтому природе всегда присуще изменчивое настроение, к тому же причудливое и при этом стабильное внутри и изменяется оно в соответствии с действующими процедурами локальной вселенной.
\vs p004 2:4 Природа --- это совершенство Рая, поделённое на неполноту [incompletion], зло и грех незавершённых вселенных. Таким образом, это частное выражает как совершенное, так и частичное, как вечное, так и временное. Продолжающаяся эволюция изменяет природу, увеличивая долю Райского совершенства и уменьшая долю зла, заблуждения и дисгармонии относительной реальности.
\vs p004 2:5 \pc Бог не присутствует лично в природе или в каких\hyp{}либо силах природы, поскольку феномен природы --- это наложение несовершенств прогрессивной эволюции, а иногда и последствий мятежного восстания, на Райские основания универсального закона Бога. В том виде, в каком она проявляется в таком мире, как Урантия, природа никогда не может быть адекватным выражением, истинным представлением, верным изображением премудрого и бесконечного Бога.
\vs p004 2:6 Природа вашего мира --- это ограничение законов совершенства эволюционными планами локальной вселенной. Какая пародия --- поклоняться природе, потому что она в узком, ограниченном смысле пронизана Богом; потому что это фаза универсальной и, следовательно, божественной силы! Природа является также проявлением незавершённого, неполного, несовершенного результата развития, роста и прогресса вселенского эксперимента в космической эволюции.
\vs p004 2:7 Видимые недостатки мира природы не указывают на какие\hyp{}то соответствующие недостатки в характере Бога. Скорее, такие наблюдаемые несовершенства являются просто неизбежными остановками в демонстрации вечно движущейся киноленты\hyp{}экранизации бесконечности. Именно эти дефекты\hyp{}прерывания совершенства\hyp{}непрерывности позволяют конечному разуму материального человека уловить мимолётный проблеск божественной реальности во времени и пространстве. Материальные проявления божественности кажутся деффективными эволюционному разуму человека только потому, что смертный человек упорно продолжает рассматривать явления природы природными глазами, используя человеческое зрение без помощи моронтийной моты\fnst{Синтетическое слово <<мота>>, возможно, происходит от латинского \bibemph{motus} (движение), потому что сверхзнание, соответствующее истинной гармонии науки, философии и религии является внутренне активным и связанным с движением, по эту сторону Рая.} или откровения, --- её компенсационный заменителя в мирах времени.
\vs p004 2:8 И природа испорчена, её прекрасное лицо покрыто шрамами, черты её иссечены восстанием, неправильным поведением, ошибочным мышлением мириад существ, которые являются частью природы, но способствуют её обезображиванию во времени. Нет, природа --- это не Бог. Природа --- не объект поклонения.
\usection{2.\bibnobreakspace НЕИЗМЕННЫЙ ХАРАКТЕР БОГА}
\vs p004 3:1 Слишком долго человек думал о Боге как о себе подобном. Бог не ревнует, не ревновал и никогда не будет ревновать человека или любого другого существа во вселенной вселенных. Зная, что Сын Создатель задумал, чтобы человек стал шедевром планетарного творения, чтобы он был правителем всей земли, вид того, как над ним господствуют его собственные низменные страсти, зрелище его преклонения перед идолами из дерева, камня, золота и эгоистичных амбиций --- эти отвратительные сцены побуждают Бога и его Сынов ревновать \bibemph{о} человеке, а не его.
\vsetoff
\vs p004 5:8 [Представлено Божественным Советником Уверсы.]
\quizlink
