\newpage
\thispagestyle{empty}
\bibmark{book}{ВВЕДЕНИЕ}
\fancyhead[C]{}
\fancyhead[C]{\bibheadfont ВВЕДЕНИЕ}

\makeatletter
\bib@raise@anchor{\bibpdfbookmark[0]{Введение}{Intr}}%
\makeatother

\begin{center}
\bibpapertitlefont
ВВЕДЕНИЕ
\end{center}

\bibbookend

Цель \bibemph{Пятого Эпохального Откровения} --- расширить космическое сознание
и улучшить качество духовного восприятия человека.
Его можно рассматривать как указание каждому смертному на его место во Вселенной и вечное предназначение.
Откровение содержит информацию о возникновении и истории нашей цивилизации, эволюции разных религий
и человеческих институтов, детальное описание жизни и учения Иисуса из Назарета,
а также приглашение к увлекательнейшему приключению --- поиску Бога.

Люди, которые уже приняли отцовство Бога по вере и вступили на путь исполнения воли небесного Отца, получат
ещё б\'ольшую возможность реализации своего космического предназначения, если отнесутся с должным вниманием
к возможности тщательного изучения данного Откровения.

Высокие идеалы и безупречный пример жизни Иисуса особенно нужны в качестве источника вдохновения сегодня,
во времена планетарного кризиса и пандемии страха и невежества,
ведущих к ослаблению и окончательной потере контакта с духом Бога Отца, пребывающим в разуме каждого нормального смертного.
Последствия подобной планетарной катастрофы трудно переоценить, ибо она эквивалентна космическому безумию и отказу от
реальности.

Среди функциональных особенностей этой книги можно привести следующие:

\begin{itemize}
\item Текст Урантийских документов основан на английском оригинале, опубликованном мною ранее в виде книги
      \bibemph{<<The British Study Edition of the Urantia Papers>>.}
      Этот текст продолжает подвергаться ревизии в соответствии с новыми открытиями фундаментальных наук, а также обнаруженными
      ошибками, допущенными при цитировании человеческих источников во время создания
      \bibemph{Пятого Эпохального Откровения} в начале XX века.
\item Замечания и текстовые варианты вынесены в сноски. Всего в данном издании \totalnfnsts\ сносок.
\item Символ \pc\ обозначает первый параграф в группе, согласно тексту первого издания 1955 г., где эти группы обозначались визуально с помощью вертикальных пробелов между параграфами.
\item Значения расстояний и температур приведены в метрических единицах, кроме тех случаев, где это было бы неуместно,
      например, <<миль Иерусема>> или идиоматических выражений типа <<идти с кем-то вторую милю>>.
\item Длинные и трудные для запоминания фразы типа <<триста сорок пять тысяч>> указаны в более компактной цифровой
      форме как <<345.000>>, а также фразы типа <<семьдесят пять процентов>> --- в виде <<75\%>>.
      Аналогично, словесные обозначения времени <<четверть пятого пополудни>> сокращены до <<16:15>>.
\item Каноническая (сверхчеловеческая) идентификация спонсора каждого документа, включая Предисловие,
      напечатана перед текстом каждого документа на отдельной строке и специальным шрифтом, а также в заголовке страницы.
\item Нумерация параграфов приводится в тексте в форме суперскрипта перед началом параграфа, а также в заголовках страниц.
\item В случаях, затруднительных для перевода, английский текст оригинала приведён в
      [квадратных скобках] в самом тексте и, где это представлялось возможным,
      добавлены пояснения и альтернативные варианты в сноске.
      Квадратные скобки использованы также для колофонов, но спутать эти два случая употребления невозможно.
\item Ссылки на литературу, использованную при создании Откровения, указаны в конце каждого документа.
\item Для удобства пользования приведён детальный \bibemph{Предметный указатель} в конце книги.
      Ссылка на номер страницы, напечатанный \textbf{жирным} шрифтом, соответствует определению данного слова.
\item Перевод осуществлялся посредством ручного редактирования чернового варианта, сгенерированного с помощью онлайн системы
      Google Translate.
\end{itemize}

Необходимость ревизии Откровения предсказана в самом его тексте \bibref[101:4.2]{p101 4:2}:
\begin{quote}
\ldots\itshape\ по прошествии нескольких лет многие из наших положений, касающихся физических наук, будут нуждаться в пересмотре вследствие дальнейшего развития науки и новых открытий.
\end{quote}
Итак, эти <<несколько лет>> уже прошли, и настало время для нас --- студентов и хранителей Откровения --- расширять и
продолжать животворящий поток \bibemph{Пятого Эпохального Откровения} --- <<Отец мой работает доныне, и я работаю>> (от Иоанна 5:17).

Понимая огромную историческую ценность этого документа,
я тщательно сохранил первое издание (1955~г.) Откровения в оригинальной форме, \bibemph{без} каких-либо изменений.
В те времена оно ещё называлось <<Книга Урантии>>.
Вместе с группой помощников мы сканировали его в сверхвысоком разрешении (1200\,dpi) и сделали результат доступным бесплатно
в печатной и электронной форме (PDF и DjVu) на моём сайте:

\begin{center}
\myurl{http://www.bibles.org.uk/guardian-plates.html}
\end{center}

Доктор Уильям Садлер отказался прояснить, каким именно образом был получен или создан текст Урантийских документов.
Тем не менее, он отчётливо разъяснил, что оно не было получено ни одним из известных
\bibemph{мистических} или \bibemph{эзотерических} методов.
В результате моих собственных исследований человеческих источников Откровения, основанных большей частью
на работах Matthew Block, я пришёл к выводу, что доктор Садлер работал
\bibemph{в партнёрстве с Богом,} и, таким образом, стал возможным синтез чисто \bibemph{человеческого} знания,
почерпнутого из многочисленных опубликованных источников, указанных в настоящем издании,
а также знания \bibemph{сверхчеловеческого,}
доступного при функционировании разума на более высоких уровнях, достигающих и соприкасающихся с надсознанием.

С тёплым чувством благодарности я хочу упомянуть имена людей, безвозмездно помогавших в работе над этим проектом
(в алфавитном порядке по фамилии): Сусанна Айвазян, Андрей Савченко, Екатерина Цупка, Ирина Чернова и Светлана Шаповалова.
В соответствии с примером, указанным в самом Откровении (см.~\bibref[73:4.4]{p074 4:4}), в работе над
\bibemph{Пятым Эпохальным Откровением} ни в коем случае не использовался ни наёмный труд, ни денежные пожертвования --- только
силы добровольцев.

\tunemarkuptwo{noquiz}{}{%
Этот PDF файл включает в себя интерактивный экзамен (\totalcurqs\ вопросов) под названием \bibemph{Космическое гражданство.}
Для пользования тестом следует кликнуть на символ \quizsymbol\ в конце любого документа или на название документа в самом тексте.
Кликая на название документа внутри теста, возвращаемся к этому документу в тексте.
}

\begin{flushleft}
\itshape
Тигран Айвазян\\
Лондон, 21 ноября 2020.\\
\end{flushleft}
