\upaper{3}{АТРИБУТЫ БОГА}
\author{Божественный Советник}
\vs p003 0:1 Бог вездесущ; Всеобщий Отец правит кр\'угом вечности. Однако в локальных вселенных он правит в лице своих Райских Сынов Создателей, так же как и дарует жизнь через этих Сынов. <<Бог дал нам жизнь вечную, и эта жизнь --- в его Сынах>>. Божьи Сыны Создатели --- это его\fnst{То есть, Бога.} личностное выражение в секторах времени и к детям планет, которые кружатся в развивающихся вселенных пространства.
\vs p003 0:2 Высоко персонализированные Сыны Бога ясно различимы более низкими категориями созданных разумных существ, и таким образом они компенсируют невидимость бесконечного и, поэтому, менее различимого Отца. Райские Сыны Создатели Всеобщего Отца --- это откровение иначе невидимого существа --- невидимого из\hyp{}за абсолютности и бесконечности, присущих кругу вечности и личностям Райских Божеств.
\vs p003 0:3 \pc Способность созидания вряд ли является атрибутом Бога; это скорее совокупность его действующей природы. И эта универсальная функция созидания проявляется вечно, поскольку она обусловлена и контролируется всеми согласованными атрибутами бесконечной и божественной реальности Первого Источника и Центра. Мы искренне сомневаемся, может ли какая\hyp{}либо одна характеристика божественной природы считаться предшествующей другим, но если бы это было так, то способность созидания в природе Божества превалировала бы над всеми остальными природами, действиями и атрибутами. И способность созидания Божества достигает кульминации во вселенской истине Отцовства Бога.
\usection{1.\bibnobreakspace ВЕЗДЕСУЩНОСТЬ БОГА}
\vs p003 1:1 Способность Всеобщего Отца присутствовать везде и одновременно составляет его вездесущность. Один только Бог может быть в двух местах, в бесчисленном множестве мест в одно и то же время. Бог одновременно присутствует <<на небе вверху и на земле внизу>>; как воскликнул Псалмопевец: <<Куда пойду от духа твоего? иль убегу куда от твоего присутствия?>>
\vs p003 1:2 <<Я --- Бог вблизи так же, как и вдали, --- говорит Господь. --- Не я ли наполняю небо и землю?>> Всеобщий Отец всё время присутствует во всех частях и во всех сердцах своего обширного творения. Он --- <<полнота того, кто наполняет всё и во всём>> и <<кто совершает всё во всём>>; и более того, концепция его личности такова, что <<небо (вселенная) и небо небес (вселенная вселенных) не могут вместить его>>. Это буквально верно, что Бог есть всё и во всём. Но даже это не есть \bibemph{весь} Бог. Бесконечный может быть окончательно раскрыт только в бесконечности; причина никогда не может быть полностью понята анализом следствий; живой Бог неизмеримо больше общей суммы творения, возникшего в результате созидательных актов его нескованной, свободной воли. Бог раскрывается через космос, но космос никогда не сможет вместить или охватить всю полноту бесконечности Бога.
\vs p003 1:3 Присутствие Отца непрерывно охраняет [patrols] главную вселенную. <<От края небес исход его и круг его до края небес; и ничто не укрыто от света его>>.
\vs p003 1:4 \pc Не только создание существует в Боге, но и Бог живёт в создании. <<Мы знаем, что пребываем в нём, ибо он живёт в нас; он дал нам свой дух. Этот дар от Райского Отца --- неразлучный спутник человека>>. <<Он есть вездесущий и всё пронизывающий собой Бог>>. <<Дух вечного Отца сокрыт в разуме каждого смертного дитя>>. <<Человек отправляется на поиски друга, в то время как этот самый друг живёт в его собственном сердце>>. <<Истинный Бог не за горами [not afar off]; он --- часть нас; его дух говорит изнутри нас>>. <<Отец живёт в дитя. Бог всегда с нами. Он есть направляющий дух вечного предназначения>>.
\vs p003 1:5 Верно было сказано о человеческом роде: <<Вы от Бога>>, ибо <<пребывающий в любви пребывает в Боге, и Бог в нём>>. Даже совершая проступок, ты мучаешь живущий в тебе дар Бога, ибо Настройщик Мыслей должен пройти через последствия злого умысла вместе с человеческим разумом, являющимся местом его заключения.
\vs p003 1:6 \pc Вездесущность Бога в действительности является частью его бесконечной природы; пространство не сосставляет преграды для Божества. Бог, в совершенстве и без ограничения, присутствует видимым образом только в Раю и в центральной вселенной. Он не присутствует таким же наблюдаемым образом в творениях, окружающих Хавону, ибо Бог ограничил своё непосредственное и явное присутствие в знак признания полновластия и божественных прерогатив равных создателей и правителей вселенных времени и пространства. Поэтому концепция божественного присутствия должна допускать широкий диапазон как способов, так и каналов проявления, включающих контуры присутствия Вечного Сына, Бесконечного Духа и Острова Рай. Также не всегда удаётся отличить присутствие Всеобщего Отца от действий вечных, равных ему существ и исполнителей, столь совершенно они исполняют все бесконечные требования его неизменного замысла. Но это не относится к личностному контуру и Настройщикам: здесь Бог действует уникально, непосредственно и исключительно.
\vs p003 1:7 \pc Вселенский Регулятор потенциально присутствует в гравитационных контурах Острова Рай во всех частях вселенной, во все времена и в одинаковой степени, в соответствии с массой, в ответ на физические потребности в его присутствии и вследствие природной сущности всего творения, которое заставляет все вещи тянуться к нему и заключаться в нём. Таким же образом Первый Источник и Центр потенциально присутствует в Безусловном Абсолюте --- вместилище несозданных вселенных вечного будущего. Так Бог потенциально пронизывает физические вселенные прошлого, настоящего и будущего. Он --- первооснова целостности так называемого материального творения. Этот недуховный потенциал Божества становится действительным здесь и там по всему уровню физических существований благодаря необъяснимому вторжению одного из его исключительных факторов на сцену вселенского действия.
\vs p003 1:8 Присутствие разума Бога [The mind presence of God] соотносится с абсолютным разумом Совместного Вершителя --- Бесконечного Духа, но в конечных творениях это присутствие лучше различимо в повсеместном функционировании космического разума Главных Духов Рая. Точно так, как Первый Источник и Центр потенциально присутствует в контурах разума Совместного Вершителя, так же он потенциально присутствует в напряжениях Всеобщего Абсолюта. Но разум человеческого типа --- это дар Дочерей Совместного Вершителя --- Божественных Помощниц эволюционирующих вселенных.
\vs p003 1:9 Присутствующий повсюду дух Всеобщего Отца координирован с функцией всеобщего присутствия духа Вечного Сына и вечным божественным потенциалом Божества Абсолюта. Но ни духовная активность Вечного Сына и его Райских Сынов, ни дарования разума Бесконечным Духом, по\hyp{}видимому, не исключают непосредственного действия Настройщиков Мыслей --- внутренних частиц Бога --- в сердцах его детей\hyp{}созданий.
\vs p003 1:10 Касательно присутствия Бога на планете, в системе, в созвездии или во вселенной, степень такого присутствия в любой единице творения есть мера степени развивающегося присутствия Верховного Существа; она определяется массовым [en masse] признанием Бога и верностью ему в обширной организации вселенной, вплоть до самих систем и планет. Поэтому иногда, в надежде сохранения и защиты этих фаз драгоценного присутствия Бога, когда некоторые планеты (или даже системы) глубоко погружаются в духовную тьму, они в определённым смысле подвергаются карантину или частичной изоляции от общения с более крупными единицами творения. И всё это, в применении к Урантии, является духовно защитной реакцией большинства миров, стремящихся, насколько это возможно, спастись от изолирующих последствий отчуждающих действий упрямого, злого [wicked] и мятежного меньшинства.
\vs p003 1:11 \pc В то время как Отец по\hyp{}родительски объединяет в контур всех своих сынов --- все личности, --- его влияние в них ограничено удалённостью их происхождения от Второго и Третьего Лиц Божества и возрастает, когда достижение ими своего предназначения приближается к этим уровням. \bibemph{Факт} присутствия Бога в разуме создания определяется пребыванием или отсутствием в нём частицы Отца, такой как Таинственный Наставник, но \bibemph{эффективность} его присутствия определяется степенью сотрудничества, оказываемого этим внутренним Настройщикам разумом их пребывания.
\vs p003 1:12 Флуктуации присутствия\fnst{То есть вариации степени его присутствия.} Отца не вызваны изменчивостью Бога. Отец не удаляется в уединение, потому что им пренебрегли; его любовь не охладевает из\hyp{}за проступка создания. Скорее, его дети, будучи наделёнными способностью выбора (относительно Его самого), при осуществлении этого выбора, непосредственно определяют степень и ограничения божественного влияния Отца в своих собственных сердцах и душах. Отец свободно подарил нам себя без ограничения и без предпочтения. Он не взирает на лица, планеты, системы или вселенные. В секторах времени он оказывает особую честь только Райским личностям Бога Семичастного --- равным создателям конечных вселенных.
\usection{2.\bibnobreakspace БЕСКОНЕЧНАЯ ВЛАСТЬ БОГА}
\vs p003 2:1 Все вселенные знают, что <<Господь Бог всемогущий царствует>>. Дела этого мира и других миров находятся под божественным надзором. <<По воле своей он действует в небесном воинстве и среди обитателей земли>>. Извечна истина: <<Нет власти не от Бога>>.
\vs p003 2:2 В пределах того, что совместимо с божественной природой, буквально истинно то, что <<с Богом всё возможно>>. Продолжительные эволюционные процессы народов, планет и вселенных находятся под совершенным контролем вселенских создателей и администраторов и разворачиваются согласно вечному замыслу Всеобщего Отца, протекая в гармонии и порядке и в соответствии с премудрым планом Бога. Есть только один законодатель. Он поддерживает миры в пространстве и обращает вселенные по бесконечному кругу вечности.
\vs p003 2:3 Из всех божественных атрибутов его всемогущество, особенно по тому, как оно доминирует в материальной вселенной, понимается лучше всего. Рассматриваемый как недуховный феномен, Бог есть энергия. Это провозглашение физического факта основано на непостижимой истине: Первый Источник и Центр --- первопричина вселенских физических явлений всего пространства. От этой божественной деятельности берёт начало вся физическая энергия и другие материальные проявления. Свет, а точнее, свет без тепла\fnst{Этот странный феномен упоминается также в \bibref[13:0.4]{p013 0:4}, \bibref[15:6.8]{p015 6:8}, \bibref[15:7.1]{p015 7:1} и \bibref[29:3.9]{p029 3:9}.}, представляет собой ещё одно из недуховных проявлений Божеств. Есть и ещё один вид недуховной энергии, почти неизвестный на Урантии и пока ещё не открытый.
\vs p003 2:4 Бог контролирует всю мощь; он проложил <<путь для молнии>>; он назначил контуры всей энергии. Он установил время и способ проявления всех видов энергии\hyp{}материи. И всё это навсегда удерживается в его непрерывном охвате --- в гравитационном контроле с центром на нижнем Рае. Таким образом, свет и энергия вечного Бога вращаются постоянно вокруг его величественного контура --- бесконечная, но упорядоченная процессия звёздных воинств, образующих вселенную вселенных. Всё творение вечно обращается вокруг Райско\hyp{}Личностного центра всех вещей и существ\fnst{А именно: вокруг Рая как центра\hyp{}источника материи, к которому вся она притягивается, и вокруг Личности (Отца) как центра\hyp{}источника личности существ, к которому они все притягиваются.}.
\vs p003 2:5 Всемогущество Отца относится к повсеместному доминированию абсолютного уровня, на котором три вида энергии --- физическая, ментальная и духовная --- неразличимы в непосредственной близости к нему, Источнику всех вещей. Разум создания, не будучи ни Райской монотой, ни Райским духом, не реагирует на Всеобщего Отца непосредственно. Бог \bibemph{настраивается} на разум несовершенства --- со смертными Урантии --- через Настройщиков Мыслей.
\vs p003 2:6 \pc Всеобщий Отец не есть преходящая сила, непостоянное могущество или флуктуирующая энергия. Могущества и мудрости Отца совершенно достаточно, чтобы справиться с любыми и всеми потребностями вселенной. Когда происходят чрезвычайные ситуации человеческого опыта, он уже предвидел их всех, и поэтому он реагирует на события вселенной не отстранённо, а согласно велениям вечной мудрости и в созвучии с требованиями бесконечного суждения. Невзирая на внешние проявления, могущество Бога не действует во вселенной как слепая сила.
\vs p003 2:7 Действительно возникают ситуации, когда кажется, что вынесены чрезвычайные постановления, приостановлены естественные законы, признаны злоупотребления, и что предпринимаются усилия для исправления ситуации; но не так обстоит дело. Такие концепции Бога берут начало в ограниченности твоей точки зрения, конечности твоего понимания и узости твоего кругозора. Такое неправильное понимание Бога --- следствие твоего глубокого невежества в отношении существования высших законов мира, величия характера Отца, бесконечности его атрибутов и факта его свободной воли.
\vs p003 2:8 Планетарные создания, в которых пребывает дух Бога, разбросанные повсюду по вселенным пространства, столь бесконечно многочисленны и разнообразны, их интеллекты так различны, их разумы так ограничены и порой столь грубы, их зрение так сужено и локализовано, что почти невозможно сформулировать обобщения закона, адекватно выражающего бесконечные атрибуты Отца и в то же время сколько\hyp{}нибудь понятного этим созданным разумным существам. Поэтому для тебя, создания, многие из деяний всесильного Создателя кажутся произвольными, отвлечёнными и нередко бессердечными и жестокими. И вновь я уверяю тебя, что это неправда. Все дела Бога целенаправленны, разумны, мудры, добры и вечно устремлены к наибольшей пользе --- не всегда отдельного существа, отдельной расы, отдельной планеты или даже отдельной вселенной; но они направлены на благополучие и достижение наибольшей пользы для всех, кого это касается, от низших до высших. В эпохах времени благополучие части может иногда казаться отличным от благополучия целого; в кругу вечности таких кажущихся различий не существует.
\vs p003 2:9 Мы все --- часть семьи Бога, и поэтому иногда нам приходится подчиняться семейной дисциплине. Многие из деяний Бога, которые так беспокоят и смущают нас, являются результатом решений и окончательных постановлений премудрости, наделяющих властью Совместного Вершителя осуществить выбор непогрешимой воли бесконечного разума, обеспечить соблюдение решений совершенной личности, чей надзор, в\'идение и забота охватывают высшее и вечное благополучие всего его громадного, широко раскинувшегося творения.
\vs p003 2:10 Таким образом, именно твоя отвлечённая, частичная, конечная, грубая и в высшей степени материалистическая точка зрения и ограничения, свойственные природе твоего существования, составляют такое препятствие, из\hyp{}за которого ты не в состоянии видеть, понимать или познавать мудрость и доброту многих божественных деяний, которые кажутся тебе исполненными столь сокрушительной жестокости, и которым, на твой взгляд, присуще столь явное безразличие к комфорту и благоденствию, к планетарному счастью и процветанию твоих собратьев\hyp{}созданий. Именно из\hyp{}за ограниченности человеческого в\'идения, именно из\hyp{}за твоего ограниченного понимания и конечного разумения ты неправильно понимаешь мотивы и искажаешь замыслы Бога. Однако в эволюционных мирах происходит много такого, что личными деяниями Всеобщего Отца не является.
\vs p003 2:11 \pc Божественное всемогущество в совершенстве согласовано с остальными атрибутами личности Бога. Могущество Бога в своём вселенском духовном проявлении обычно ограничено только тремя условиями или ситуациями:
\vs p003 2:12 \ublistelem{1.}\bibnobreakspace Природой Бога, особенно его бесконечной любовью, а также истиной, красотой и добротой.
\vs p003 2:13 \ublistelem{2.}\bibnobreakspace Волей Бога, его милосердной помощью и отеческим отношением к личностям вселенной.
\vs p003 2:14 \ublistelem{3.}\bibnobreakspace Законом Бога, праведностью и правосудием вечной Райской Троицы.
\vs p003 2:15 \pc Бог неограничен в могуществе, божественен по природе, окончателен в воле, бесконечен в атрибутах, вечен в мудрости и абсолютен в реальности. Но все эти черты Всеобщего Отца объединены в Божестве и универсально выражены в Райской Троице и в божественных Сынах Троицы. В остальном, вне Рая и центральной вселенной Хавоны, всё относящееся к Богу ограничено эволюционным присутствием Верховного, обусловлено возникающим присутствием Предельного и координировано тремя экзистенциальными Абсолютами: Божеством, Всеобщим и Безусловным. И присутствие Бога ограничено таким образом потому, что такова воля Бога.
\usection{3.\bibnobreakspace ВСЕЗНАНИЕ БОГА}
\vs p003 3:1 <<Бог знает всё>>. Божественный разум сознаёт и хорошо знает мысль всего творения. Его знание событий универсально и совершенно. Исходящие из него божественные сущности являются частью его; тот, кто <<удерживает в равновесии облака>>, является также <<совершенным в знании>>. <<На всяком месте очи Господни>>. Так сказал ваш великий учитель о незначительном воробье: <<Ни один из них не упадёт на землю без ведома моего Отца>>, а также: <<Даже волосы на голове вашей сосчитаны>>. <<Он исчисляет число звёзд; он называет их всех по имени>>.
\vs p003 3:2 Всеобщий Отец --- единственная личность во всей вселенной, которая действительно знает число звёзд и планет в космосе. Все миры каждой вселенной постоянно в сознании Бога. Он говорит также: <<Я увидел страдание народа моего, услышал вопль его, знаю скорби его>>. Ибо <<Господь смотрит с небес; он видит всех сынов человеческих; с места, где обитает, он смотрит на всех жителей земли>>. Каждое сотворённое дитя может действительно сказать: <<Он знает путь, которым я иду, и когда испытает меня, выйду, как золото>>. <<Бог знает, когда мы садимся и когда встаём; он понимает наши мысли издалека, и все пути наши известны ему>>. <<Всё обнажено и открыто перед глазами того, кому должны дать отчёт>>. И для каждого человеческого существа настоящим утешением должно стать понимание того, что <<он знает твою конструкцию; он помнит, что ты --- прах>>. Говоря о живом Боге, Иисус сказал: <<Отец ваш знает, в чём вы имеете нужду, прежде вашего прошения у него>>.
\vs p003 3:3 Бог имеет неограниченную власть знать всё; его сознание универсально. Его личностный контур охватывает все личности, и его знания даже о низших созданиях дополняются косвенно через нисходящую серию божественных Сынов и напрямую через внутренних Настройщиков Мыслей. И кроме того, Бесконечный Дух постоянно повсюду присутствует.
\vs p003 3:4 Мы не до конца уверены в том, выбирает ли Бог знать заранее о событиях совершения греха или нет. Но даже если Бог предвидит добровольные поступки своих детей, такое предвидение ни в коей мере не отменяет их свободы. В одном нет сомнения: Бог никогда не подвержен неожиданности.
\vs p003 3:5 \pc Всемогущество не подразумевает способности совершать невыполнимое, неподобающее Богу действие. Так же и всеведение не предполагает знания непознаваемого. Но такие утверждения вряд ли можно сделать понятными конечному разуму. Создание едва ли может понять область и ограничения воли Создателя.
\usection{4.\bibnobreakspace БЕСПРЕДЕЛЬНОСТЬ БОГА}
\vs p003 4:1 Последовательные дарования себя вселенным по мере их появления ни в коей мере не уменьшают потенциал мощи или запас мудрости, поскольку они продолжают пребывать и покоиться в центральной личности Божества. В потенциале силы, мудрости и любви Отец никогда не уменьшал ничего из того, что имел и не лишался каких\hyp{}либо атрибутов своей прекрасной личности в результате неограниченного посвящения себя Райским Сынам, своим подчинённым творениям и их разнообразным созданиям.
\vs p003 4:2 Создание каждой новой вселенной требует новой поправки к гравитации; но даже если творение будет продолжаться неопределённо долго, вечно и даже до бесконечности, так что в конечном итоге материальное творение будет существовать без ограничений, то и тогда мощь контроля и координации, заключённая в Острове Рай, окажется достаточной и адекватной для управления, контроля и координации такой бесконечной вселенной. И после этого дарования неограниченной силы и мощи безграничной вселенной, Бесконечный всё равно будет в той же степени заряжен силой и энергией; Безусловный Абсолют всё равно не уменьшится; Бог по\hyp{}прежнему будет обладать всё тем же бесконечным потенциалом --- точно так, как если бы сила, энергия и мощь никогда и не изливались для наделения всё новых и новых вселенных.
\vs p003 4:3 То же самое и с мудростью: тот факт, что разум так щедро распределяется для мышления в мирах, никоим образом не истощает центральный источник божественной мудрости. По мере того как вселенные умножаются, а существа миров возрастают числом до невообразимых пределов, если разум и продолжит без конца дароваться этим существам высокого и низкого уровня, центральная личность Бога всё ещё будет продолжать содержать в себе тот же самый вечный, бесконечный и премудрый разум.
\vs p003 4:4 Тот факт, что он от себя посылает духовных посланников, чтобы поселиться в мужчинах и женщинах твоего мира и других миров, никоим образом не ослабляет его способности действовать как божественная и всемогущая духовная личность; и нет абсолютно никакого предела мере или числу таких духов\hyp{}Наставников, которых он в состоянии и может послать. Это дарение себя своим созданиям создаёт безграничную, почти немыслимую будущую возможность прогрессивных и последовательных [progressive and successive] существований для этих божественно наделённых смертных. И такое щедрое распределение себя в виде этих помогающих духовных сущностей никоим образом не умаляет мудрости и совершенства истины и знания, которые покоятся в личности премудрого, всезнающего и всемогущего Отца.
\vs p003 4:5 \pc Для смертных времени есть будущее, но Бог обитает в вечности. Несмотря на то, что я происхожу почти из того места, где обитает Божество, я, тем не менее, не смею предполагать, что говорю с совершенством понимания о бесконечности многих из божественных атрибутов. Только бесконечность разума может полностью постичь бесконечность существования и вечность действия.
\vs p003 4:6 \pc Смертный человек никак не может познать бесконечность небесного Отца. Конечный разум не может до конца продумать такую абсолютную истину или факт. Однако то же самое конечное человеческое существо действительно может \bibemph{почувствовать} --- буквально испытать --- полное и неизменное воздействие такой бесконечной ЛЮБВИ Отца. Такую любовь можно по\hyp{}настоящему испытать, и хотя качество опыта безгранично, количество такого опыта строго ограничено человеческой способностью к духовной восприимчивости и связанной с этим способностью любить Отца в ответ.
\vs p003 4:7 Конечная оценка бесконечных качеств далеко превосходит логически ограниченные способности создания из\hyp{}за того факта, что смертный человек создан по образу Бога --- внутри него живёт фрагмент бесконечности. Поэтому кратчайший и наилучший путь [nearest and dearest approach] человека к Богу --- это подход через любовь и посредством любви, ибо Бог есть любовь. И вся совокупность таких уникальных отношений --- это реальный опыт в области космической социологии, связь между Создателем и созданием --- любовь между Отцом и ребёнком.
\usection{5.\bibnobreakspace ВЕРХОВНОЕ ПРАВЛЕНИЕ ОТЦА}
\vs p003 5:1 В своём контакте с пост\hyp{}Хавонскими творениями Всеобщий Отец использует своё бесконечное могущество и окончательную власть не путём прямой передачи, а через своих Сынов и подчинённых им личностей. И всё это Бог делает по своей собственной свободной воле. Любые и все делегированные полномочия, если возникнет необходимость, если это станет выбором божественного разума, могут быть задействованы напрямую; но, как правило, такое действие имеет место только в результате неспособности уполномоченной личности оправдать божественное доверие. В таких случаях и перед лицом подобной провинности, в пределах зарезервированной божественной мощи и потенциала, Отец действует независимо и в соответствии с указаниями своего собственного выбора; и этот выбор всегда непогрешимо совершенный и бесконечно мудрый.
\vs p003 5:2 Отец правит через своих Сынов; на всём протяжении вселенской организации существует непрерывная цепь правителей, заканчивающаяся Планетарными Принцами, которые направляют судьбы эволюционных сфер необъятных владений Отца. Следующие слова --- не просто поэтическое выражение: <<Господня земля, и чт\'о наполняет её>>. <<Он низлагает царей и ставит царей>>. <<Всевышние правят в царствах людей>>.
\vs p003 5:3 В делах человеческих сердец Всеобщий Отец не всегда может добиваться своего; но в руководстве и судьбе планеты побеждает божественный план; вечный замысел мудрости и любви торжествует.
\vs p003 5:4 Иисус сказал: <<Мой Отец, который дал их мне, больше всех; и никто не может похитить их из руки моего Отца>>. Окидывая взором многообразные произведения и глядя на ошеломляющую грандиозность почти беспредельного творения Бога, ты можешь поколебаться в твоей концепции его первичности, но ты не должен отказываться от того, чтобы принять его как надёжно и вечно восседающего на троне в Райском центре всех вещей и как благодетельного Отца всех разумных существ. Есть только <<один Бог и Отец всех, который над всеми и во всех>>, <<и он прежде всего, и в нём заключается всё>>.
\vs p003 5:5 \pc Неопределённости жизни и превратности существования никоим образом не противоречат концепции универсального владычества Бога. Вся жизнь эволюционирующего существа окружена определёнными \bibemph{неизбежностями.} Вдумайся в следующее:
\vs p003 5:6 \ublistelem{1.}\bibnobreakspace Желательна ли \bibemph{храбрость} --- сила характера? Тогда человек должен воспитываться в среде, которая требует преодоления трудностей и реагирования на разочарования.
\vs p003 5:7 \ublistelem{2.}\bibnobreakspace Желателен ли \bibemph{альтруизм} --- служение ближним? Тогда жизненный опыт должен обеспечивать столкновения с ситуациями социального неравенства.
\vs p003 5:8 \ublistelem{3.}\bibnobreakspace Желательна ли \bibemph{надежда} --- величие доверия? Тогда человеческое существование должно постоянно сталкиваться с незащищённостью и периодическими сомнениями.
\vs p003 5:9 \ublistelem{4.}\bibnobreakspace Желательна ли \bibemph{вера} --- верховное утверждение человеческой мысли? Тогда разум человека должен попадать в то затруднительное положение, где он всегда знает меньше, чем он может верить.
\vs p003 5:10 \ublistelem{5.}\bibnobreakspace Желательна ли \bibemph{любовь к истине} и готовность идти, куда бы онa ни велa? Тогда человек должен вырастать в мире, где присутствует заблуждение, и всегда возможна ложь.
\vs p003 5:11 \ublistelem{6.}\bibnobreakspace Желателен ли \bibemph{идеализм} --- приближающаяся к истине концепция божественного\fnst{Или концепция, приближающая к божественному, (англ. the approaching concept of the divine).}? Тогда человек должен бороться в среде относительной доброты и красоты, в окружении, стимулирующем неудержимое стремление к лучшему.
\vs p003 5:12 \ublistelem{7.}\bibnobreakspace Желательна ли \bibemph{верность} \bibemph{[loyalty]} --- преданность высшему долгу? Тогда человек должен продолжать действовать среди возможностей предательства и дезертирства. Доблесть в преданности долгу заключается в предполагаемой опасности невыполнения обязательств [default].
\vs p003 5:13 \ublistelem{8.}\bibnobreakspace Желательно ли \bibemph{бескорыстие} --- дух самозабвения [self\hyp{}forgetfulness]? Тогда смертный человек должен жить лицом к лицу с несмолкаемым криком собственного <<я>>, от которого не уйти, требующим признания и почестей. Человек не мог бы активно избирать божественную жизнь, если бы не было никакой эгоистичной жизни, чтобы от неё отказаться. Человек никогда не смог бы ухватиться за спасительную праведность, если бы не было потенциального зла, чтобы по контрасту с ним возвышать и отделять от него добро.
\vs p003 5:14 \ublistelem{9.}\bibnobreakspace Желательно ли \bibemph{удовольствие} --- удовлетворение, дающее счастье? Тогда человек должен жить в мире, где альтернатива боли и вероятность страдания --- это постоянно присутствующие возможности опыта.
\vs p003 5:15 \pc По всей вселенной любая единица рассматривается как часть целого. Выживание части зависит от сотрудничества с планом и замыслом целого, от искреннего желания и совершенной готовности исполнять божественную волю Отца. Единственный эволюционный мир без ошибок (возможности неразумного суждения) --- это мир без \bibemph{свободного} разума. Во вселенной Хавона есть миллиард совершенных миров с их совершенными обитателями, но развивающийся человек должен быть подвержен ошибкам, если он хочет быть свободным. Свободный и неопытный разум никак не может поначалу быть однородно мудрым. Возможность ошибочного суждения (зло) становится грехом только тогда, когда человеческая воля сознательно подтверждает и преднамеренно принимает заведомо безнравственное суждение.
\vs p003 5:16 \pc Полное понимание истины, красоты и доброты присуще совершенству божественной вселенной. Обитатели миров Хавоны не нуждаются в существовании относительных уровней ценности [relative value levels] в качестве стимула для выбора; такие совершенные существа способны идентифицировать и выбирать добро при отсутствии всех контрастных и заставляющих думать моральных ситуаций. Но все такие совершенные существа, по нравственной природе и духовному статусу, таковы в силу факта существования. Они эмпирически заслужили продвижения только в пределах их врождённного статуса. Смертный же человек даже свой статус кандидата на восхождение должен заслужить своей собственной верой и надеждой. Всё божественное, что улавливается человеческим разумом и обретается человеческой душой, является эмпирическим достижением; это --- \bibemph{реальность} личного опыта и, следовательно, уникальное владение, в отличие от врождённой доброты и праведности непогрешимых личностей Хавоны.
\vs p003 5:17 \pc Создания Хавоны от природы храбры, но не отважны в человеческом смысле. Они врождённо добры и внимательны, но едва ли альтруистичны, по\hyp{}человечески. Они ожидают приятного будущего, но не полны надежды с той остротой, которая свойственна доверчивым смертным эволюционных сфер, лишённых определённости. Они верят в стабильность вселенной, но совершенно чужды той спасительной веры, посредством которой смертный человек поднимается от статуса животного до врат Рая. Они любят истину, но ничего не знают о её душеспасительных качествах. Они --- идеалисты, но такими и родились; они совершенно не знают об экстазе становления таковыми благодаря пьянящему [exhilarating] выбору. Они верны, но никогда не испытывали трепета от искренней и сознательной преданности долгу перед лицом искушения этот долг преступить. Они бескорыстны, но никогда не достигали этих уровней опыта благодаря величественной победе над воинствующим <<я>>. Они наслаждаются удовольствием, но не понимают сладости удовольствия при избавлении от возможности боли.
\usection{6.\bibnobreakspace ПЕРВИЧНОСТЬ ОТЦА}
\vs p003 6:1 С божественным бескорыстием, непревзойдённой щедростью Всеобщий Отец уступает власть и делегирует\fnst{То есть <<передаёт часть своих полномочий>>.} могущество, но он по\hyp{}прежнему первичен; его рука на мощном рычаге обстоятельств вселенских сфер; он сохранил за собой все окончательные решения и безошибочно держит всемогущий скипетр вето своего вечного замысла с неоспоримой властью над благополучием и предназначением раскинувшегося, кружащегося и описывающего вечный круг творения.
\vs p003 6:2 Верховная власть Бога безгранична; это фундаментальный факт всего творения. Вселенная не была неизбежной. Вселенная не случайна и не существует сама по себе. Вселенная --- результат творения, и поэтому полностью подчиняется воле Создателя. Воля Бога --- это божественная истина, живая любовь; поэтому совершенствующиеся творения эволюционных вселенных характеризуются как добром --- близостью к божественности, так и потенциальным злом --- удалённостью от божественности.
\vs p003 6:3 \pc Всякая религиозная философия рано или поздно приходит к концепции объединённого вселенского правления, единого Бога. Вселенские причины не могут быть ниже вселенских следствий. Источник потоков вселенской жизни и космического разума должен быть выше уровней их проявления. Человеческий разум невозможно последовательно объяснить в терминах более низких форм существования. Разум человека можно по\hyp{}настоящему постичь, только признав реальность более высоких уровней мысли и целенаправленной воли. Человек как нравственное существо необъясним, пока не признана реальность Всеобщего Отца.
\vs p003 6:4 Философ\hyp{}механист\fnst{Приверженец \bibemph{механицизма} --- миропонимания, рассматривающего мир как механизм, противопоставлялся \bibemph{витализму}.} утверждает, что отвергает идею универсальной и суверенной воли, той самой суверенной воли, деятельность которой в разработке законов вселенной он так глубоко почитает. Какое непреднамеренное почтение механист воздаёт закону\hyp{}Создателю\fnst{То есть \bibemph{закону,} с ошибочной точки зрения механиста, но, на самом деле, \bibemph{Создателю.}}, когда считает, что такие законы действуют сами по себе и не требуют объяснений!
\vs p003 6:5 Очеловечивать Бога --- это огромная ошибка, за исключением концепции внутреннего Настройщика Мыслей, но даже это не настолько глупо, как попытка полностью \bibemph{механизировать} идею Первого Великого Источника и Центра.
\vs p003 6:6 \pc Страдает ли Райский Отец? Я не знаю. Сыны Создатели, безусловно, могут страдать и иногда страдают так же, как смертные. Вечный Сын и Бесконечный Дух страдают в модифицированном\fnst{То есть в некотором ином, определённом.} смысле. Я думаю, что Всеобщий Отец страдает, но я не могу понять \bibemph{как;} возможно, через личностный контур или индивидуальность Настройщиков Мыслей и другие посвящения его вечной природы. О смертных расах он сказал: <<Во всех скорбях ваших я скорблю>>. Он, несомненно, испытывает отеческое и сочувственное понимание; он может действительно страдать, но я не понимаю природы этого.
\vs p003 6:7 \pc Бесконечный и вечный Правитель вселенной вселенных --- это мощь, форма, энергия, процесс, образец, принцип, присутствие и идеализированная реальность. Но он больше; он --- личностный; он проявляет суверенную волю, испытывает самосознание божественности, исполняет указы творческого разума, стремится к удовлетворению от реализации вечного замысла и проявляет любовь и привязанность Отца к своим вселенским детям. И все эти более личностные черты Отца можно лучше понять, наблюдая их такими, какими они были раскрыты в посвященческой жизни Михаила, вашего Сына Создателя, когда он жил во плоти на Урантии.
\vs p003 6:8 \pc Бог Отец любит людей; Бог Сын служит людям; Бог Дух вдохновляет детей вселенной на бесконечно восходящее приключение поиска Бога Отца путями, предопределёнными Богом Сыновьями\fnst{Необычная конструкция, но это дословный перевод (англ. by God the Sons).} через милосердное служение [ministry of the grace] Бога Духа.
\vsetoff
\vs p003 6:9 [Являясь Божественным Советником, назначенным для представления откровения Всеобщего Отца, я продолжил своё дело данным изложением атрибутов Божества.]
\quizlink
\begin{thebibliography}{100}
\bibitem{Knudson1}
Alber C. Knudson.
{<<The Doctrine of God>>.}
{\em New York: Abingdon-Cokesbury Press}, 1930.
\bibitem{Hume1}
Robert Ernest Hume, M.A., Ph.D.,
{<<Treasure\hyp{}House of the Living Religions: Selections from Their Sacred Scriptures>>.}
{\em New York: Charles Scribner's Sons}, 1932.
\bibitem{Lewis1}
Edwin Lewis
{<<A Plea for the Reality, Adequacy and Availability of God>>.}
{\em New York: The Abingdon Press.}, 1931.
\end{thebibliography}
